% Setup -------------------------------

\documentclass[a4paper]{report}
\usepackage[a4paper, total={6in, 10in}]{geometry}
\setcounter{secnumdepth}{3}
\setcounter{tocdepth}{3}

\PassOptionsToPackage{hyphens}{url}
\usepackage{hyperref}
\usepackage{indentfirst}

% Encoding
%--------------------------------------
\usepackage[T1]{fontenc}
\usepackage[utf8]{inputenc}
%--------------------------------------

% Portuguese-specific commands
%--------------------------------------
\usepackage[portuguese]{babel}
%--------------------------------------

% Hyphenation rules
%--------------------------------------
\usepackage{hyphenat}
%--------------------------------------

% Capa do relatório

\title{
	Desenvolvimento de Aplicações Web
	\\ \Large{\textbf{Trabalho Prático}}
	\\ -
	\\ Mestrado em Engenharia Informática
	\\ \large{Universidade do Minho}
	\\ Relatório
}
\author{
	\begin{tabular}{ll}
		\textbf{Grupo}
		\\\hline
		PG41080 & João Ribeiro Imperadeiro
		\\
		PG41081 & José Alberto Martins Boticas
		\\
		PG41091 & Nelson José Dias Teixeira
	\end{tabular}
}

\date{\today}

\begin{document}

\begin{titlepage}
    \maketitle
\end{titlepage}

% Índice

\tableofcontents

% Introdução

\chapter{Introdução} \label{intro} {
	Neste trabalho prático é requerido o desenvolvimento de uma rede social de alunos para alunos onde possam partilhar materiais, discutir datas, combinar eventos, entre outros. 
	
    Os requisitos mínimos para este projeto são uma plataforma que:
    \begin{itemize}
        \item faça a gestão de utilizadores com autenticação;
        \item faça distinção do que é público e privado;
        \item permita aos utilizadores publicar conteúdos;
        \item permita partilhar ficheiros ou fotos;
    \end{itemize}

    É necessário ainda assegurar que só utilizadores autenticados poderão publicar ou comentar publicações de outros.

    Neste projeto foram cumpridos estes requisitos e ainda se adicionaram mais alguns.
}

\chapter{A plataforma}
	\section{Informatics Social Network (ISN)}

	O sistema é dividido em duas partes principais: servidor de API e servidor de interface.

    O servidor de API trata de toda a interação com a base de dados, servindo de \textit{middleware} entre a mesma e o servidor de interface.

	O servidor de interface trata da apresentação das informações ao utilizador, sendo que, para isso, utiliza ficheiros \texttt{pug}, que geram \texttt{HTML}, aliados a \texttt{CSS}.
    As informações são obtidas recorrendo ao servidor de API.

	\section{Funcionalidades}
    De seguida, apresentam-se as principais funcionalidades do sistema.
	
	\subsection{Autenticação}
    De forma a poder usufruir da plataforma, um utilizador terá de se autenticar. Para isso, usará um email e uma \textit{password}, caso já se encontre registado.
    Caso contrário, deverá registrar-se, indicando os dados pedidos pela rede social.

    Aquando de um novo registo, as informações respetivas são enviadas do servidor de interface para o servidor de API, com esta última a ser responsável por guardar as mesmas na base de dados.


	\subsection{Publicações}
    Um utilizador pode publicar conteúdos. Assim, poderá escrever uma mensagem, à sua escolha, e ainda adicionar ficheiros, no número que entender. 
    Para além disso, pode ainda escolher se essa mesma publicação poderá ser vista por qualquer pessoa, apenas pelos seus amigos ou pelos membros de um certo grupo.

    Uma publicação é capaz de identificar e guardar as tags utilizadas na mesma, iniciadas por "#".


    \subsection{Grupos}
    Um grupo pode ser criado por qualquer utilizador e pode ser público (qualquer pessoa pode aceder livremente ao mesmo e aos seus conteúdos) ou privado (apenas os seus membros podem aceder aos conteúdos partilhados no mesmo).

    A página de um grupo apresenta três abas: as publicações do mesmo; os seus membros; os ficheiros partilhados em publicações no mesmo.


    \subsection{Eventos}
    Um evento pode ser criado por qualquer utilizador.

    A página de um evento apresenta duas abas: as informações do mesmo (título e data, por exemplo); os seus participantes.


    \subsection{Informação de perfil}
    É possível visitar a página do perfil de um qualquer utilizador. Nesta são exibidas todas as informações desse utilizador, como o seu nome, email, data de nascimento, biografia, amigos e publicações.
    Caso esta página seja a do próprio utilizador, é ainda disponibilizado um botão "Editar". Este redireciona a mesma para um outra página que permite editar as informações.


    \subsection{Amigos}
    Um utilizador pode adicionar e remover amigos, sendo que uma adição fica dependente da aceitação do outro utilizador envolvido.

    É disponibilizada uma página com os pedidos de amizade que um utilizador recebe, na qual o mesmo pode aceitá-los ou recusá-los.


    \subsection{Notificações}
    Um utilizador receberá notificações sempre que:
    \begin{itemize}
        \item um amigo faz uma publicação
        \item é feita uma publicação num grupo ao qual o utilizador pertence
        \item um outro utilizador se junta a um grupo no qual o utilizador é membro
        \item um outro utilizador se junta a um evento no qual o utilizador participará
        \item um outro utilizador aceita um pedido de amizade do utilizador 
    \end{itemize}


	\section{Arranque do sistema} % A COMPLETAR
	Vejamos como colocar a plataforma em funcionamento:

	\begin{enumerate}
        \item Executa-se o comando \texttt{mongoimport }
              Pode ainda ser necessário iniciar o MongoDB, com o comando \texttt{mongod}.

		\item Executa-se o servidor de API, fazendo \texttt{npm i} seguido de \texttt{npm start}.
		
		\item executa-se o catálogo, fazendo \textit{make runCatalog};
	
		\item executa-se um negociador (ou vários, tendo para isso de alterar o ficheiro do servidor), fazendo \textit{make compileJava} e \textit{make runDealer};
	
		\item colocam-se os clientes a correr, ao executar \textit{make runClient}.
	\end{enumerate}

\chapter{Conclusão}
Este trabalho permitiu a aplicação dos conteúdos lecionados ao longo da UC, bem como o aprofundamento e alargamento dos conhecimentos relativos ao desenvolvimento de aplicações web.


\chapter{Webgrafia}
	\begin{itemize}
		\item \textbf{\textit{W3}}:
		\par \textit{\url{https://w3schools.com}}
        \item \textbf{Documentação - \textit{Bootstrap}}:
		\par \textit{\url{https://getbootstrap.com/docs/4.4/getting-started/introduction/}}
		\item \textbf{Documentação - \textit{Express}}:
		\par \textit{\url{https://expressjs.com/en/5x/api.html}}
        \item \textbf{\textit{Pug}}:
		\par \textit{\url{https://pugjs.org/api/getting-started.html}}
        \item \textbf{Documentação - \textit{MongoDB}}:
		\par \textit{\url{https://docs.mongodb.com/}}
    \end{itemize}

\end{document}